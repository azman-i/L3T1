\documentclass{article}
\usepackage[utf8]{inputenc}
\usepackage{subfiles}
\usepackage{lipsum}
\usepackage[algoruled, linesnumbered]{algorithm2e}
\usepackage{listings}
\usepackage{color}
\usepackage{tikz}
\usetikzlibrary{ar

\definecolor{codegreen}{rgb}{0,0.6,0}
\definecolor{codegray}{rgb}{0.5,0.5,0.5}
\definecolor{codepurple}{rgb}{0.58,0,0.82}
\definecolor{backcolour}{rgb}{1,1,0}

\lstdefinestyle{mystyle}{
    backgroundcolor=\color{backcolour},   
    commentstyle=\color{codegreen},
    keywordstyle=\color{magenta},
    numberstyle=\tiny\color{codegray},
    stringstyle=\color{blue},
    basicstyle=\footnotesize,
    breakatwhitespace=false,         
    breaklines=true,                 
    captionpos=b,                    
    keepspaces=true,                 
    numbers=left,                    
    numbersep=5pt,                  
    showspaces=false,                
    showstringspaces=false,
    showtabs=false,                  
    tabsize=4
}

\lstset{style=mystyle}

\title{Pthyagorean Theorem}
\author{1505063}
\date{\today}

\begin{document}

\maketitle

\section{Introduction}
In this document, we present the very famous theorem in mathematics: \textit{Pythagorean
theorem}, which is stated as follows.\newline
\textbf{Theorem 1.1 (Pythagorean theorem)}\textit{ The square of the hypotenuse (the
side opposite the right angle) is equal to the sum of the squares of the other two
sides.}\newline
\hspace{10mm} Numerous  mathematicians proposed various proofs to the theorem. The
theorem was long known even before the time of Pythagoras. Pythagoras was
the first to provide with a sound proof. The proof that Pythagoras gave was
by \textit{rearrangement}. Even the great Albert Einstein also proved the theorem
without rearrangement, rather by using dissection. Figure 1 shows the visual
representation of the theorem.
\subfile{pictures.tex}
\section{Trigonometric forms}
Lots of other forms of the same theorem exist. The most useful, perhaps, are expressed in trigonometric terms, as follows:
\begin{equation}
\label{eq}
1 = \cos^2 \theta + \sin^2 \theta;
\end{equation}
\begin{equation}
1 = \sec^2 \theta - \tan^2 \theta;
\end{equation}

\begin{equation}
 \cos^2 {\theta}-\cot^2 \theta=1;
\end{equation}

\subsection{Representing The First}
Taking \ref{eq}, we can show them as shown in Figure2. When we take a point at
unit distance from the origin, the y and x co-ordinates become $\sin\theta$ and $\cos\theta$
respectively. Therefore, sum of the squares of the two becomes equal to the
square of the unit distance, which of course, is 1.


\end{document}