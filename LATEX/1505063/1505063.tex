\documentclass{article}
\usepackage[utf8]{inputenc}
\usepackage{lipsum}
\usepackage{color}
\usepackage{amsmath}
\usepackage{graphicx}
\title{Golden Ratio}

\begin{document}

\begin{titlepage}

    \begin{center}
        \vspace*{1cm}
        
        \Huge
        \textbf{Golden Ratio}
        
        \vspace{0.5cm}
        \LARGE
        CSE 300
        
        %\vspace{1.5cm}
        
        \textbf{Course Name:Technical Writing and Representation}
        
     
        
       
        
        \vspace{2 cm}
         
        \includegraphics[width=0.4\textwidth]{image/logoBIRN}
        \vfill
        \Large
         
        Khairul Azman\\
        Std Id:1505063\\
          Department of Computer Science and Engineering\\
          Bangladesh University of Engineering and Technology\\
      \today
        
    \end{center}
\end{titlepage}

\tableofcontents
\newpage
\section{Definition}
\label{sec:definition}
 The Golden ratio is a special number found by dividing a line into two parts so that the longer part divided by the smaller part is also equal to the whole length divided by the longer part. It is often symbolized using phi, after the 21st letter of the Greek alphabet.
 \section{History}
Throughout history, the ratio for length to width of rectangles of 1.61803 39887 49894 84820 has been considered the most pleasing to the eye. This ratio was named the golden ratio by the Greeks. In the world of mathematics, the numeric value is called "phi", named for the Greek sculptor Phidias. The space between the columns form golden rectangles. There are golden rectangles throughout this structure which is found in Athens, Greece.

\begin{figure}[h]
     \centering
      \includegraphics[width=0.5\textwidth]{image/golden.jpg}
     \caption{Structure in Greek}
     \label{fig:bl}
 \end{figure}
\section{Derivation of Golden Ratio}
\subsection{Proof}
From \ref{sec:definition} we found that golden ratio is longer part divided by the smaller part.
\begin{figure}[h]
     \centering
      \includegraphics[width=0.3\textwidth]{image/derivation.png}
     \caption{Golden ratio}
     \label{fig:bl}
 \end{figure}
 \newline So in figure \ref{fig:bl} the smaller part is $b$ and the longer part is $a$.So Golden ratio is 
 \begin{equation}
     \label{eqn:eq}
     \varphi=\frac{a}{b}
 \end{equation}
 from definition of golden ratio
 \begin{equation}
     \frac{a+b}{a}=\frac{a}{b}
 \end{equation}
 \begin{equation*}
    \implies 1+\frac{b}{a}=\frac{a}{b}
 \end{equation*}
 From equation \ref{eqn:eq} we find that 
 \begin{equation*}
     1+1/\varphi=\varphi;
 \end{equation*}
     
It is a two degree polynomial equation.By solving it we find \newline
\begin{equation*}
    \varphi=\frac{1+\sqrt{5}}{2}
\end{equation*}
which means that \newline
\begin{equation*}
    \varphi=1.618033
\end{equation*}
\subsection{Why Golden Ratio is 1.618033?}
Throughout  history, the ratio for length to width of rectangles of 1.61803399 has been considered the most pleasing to the eye. This ratio  was named the golden ratio by the Greeks.\newline
It's called golden ratio cause it is the divine propo pentagon, pentagram, decagon and dodecahedron. 
\section{Application}
 \begin{enumerate}
     \item Architecture
     \item Logos ans Trademarks
     \item Market Analysis
   \item  Plastic surgery
   \item Fractals
   \item Nature
 \end{enumerate}
 
     
\end{document}